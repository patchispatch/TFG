\documentclass[10pt, a4paper]{aqademic}

\usepackage[spanish]{babel}
\selectlanguage{spanish}

\usepackage[default]{sourcesanspro}
\usepackage[T1]{fontenc}

% Document packages

\usepackage{amsmath}
\usepackage{amsfonts}
%\usepackage[type=CC, modifier=by-nc-sa, version=4.0]{doclicense}
\usepackage{graphicx}
\graphicspath{{img/}}
\usepackage[numbers]{natbib}

% Document settings

\author{Juan Ocaña Valenzuela}
\title{Trabajo Fin de Grado}

\AqSetChapter{Capítulo}

% Document composition

\begin{document}

\AqMaketitle[%
	cover    = img/logo.png,
	org      = Grado en Ingeniería Informática,
	subtitle = Aplicación de gestión de rutinas semanales basada en tecnologías web
]
\tableofcontents

% Capítulo: introducción
%	1. Resumen
%	2. Resumen en inglés (mínimo 500 palabras)
\chapter{Introducción}

\textbf{Palabras clave:} palabras clave

\section{Resumen}

\textbf{NOTA: provisional}

Existe una gran oferta en cuanto a herramientas de organización personal. Desde activas comunidades que apuestan por formatos tradicionales como agendas y calendarios personalizados hasta numerosos servicios preparados para ajustarse a cualquier tipo de persona, escoger un método de organización puede suponer una odisea para alguien interesado en la materia. 

\medskip

Muchas de las herramientas más populares en este campo optan por ofrecer un servicio de suscripción ---a menudo con versión gratuita--- con ventajas tales como sincronización en la nube o integración con servicios adicionales. Sin embargo, a veces estas opciones son demasiado complejas para algunas personas, que simplemente buscan algo más sencillo o acorde a sus necesidades.

\medskip

Se plantea una propuesta de aplicación con una funcionalidad simple, orientada a reforzar y asistir en la creación de rutinas y hábitos, utilizando como soporte un sistema de objetivos y una estructura de horario semanal. La idea principal es brindar una funcionalidad simple a la vez que intuitiva, haciendo énfasis en la accesibilidad y facilidad de uso, optando por no incluir herramientas complejas, tales como un sistema de calendario o un bloc de notas ---opciones que muchos de los servicios mencionados anteriormente implementan---.

\medskip

La aplicación propuesta se ha implementado utilizando tecnologías web muy extendidas y con gran soporte en la actualidad, y se acoge a los estándares de accesibilidad WAI-ARIA definidos por el \textit{World Wide Web Consortium}.

\newpage


\textbf{Keywords:} keywords


\section{\textit{Abstract}}

The personal organization market offers a lot of options and alternatives. From huge online communities using and reimagining the traditional notebooks and diaries


% Capítulo: estado del arte
%	1. Crítica al estado del arte
%	2. Propuesta
\chapter{Estado del arte}

No sé si este capítulo procede en esta modalidad de TFG.

\section{Crítica al estado del arte}

\section{Propuesta}


% Capítulo: análisis de objetivos y metodología
%	1. Metodología de desarrollo
%	2. Análisis de requisitos
%	3. Análisis de tecnologías
%	4. Presupuesto
\chapter{Análisis de objetivos y metodología}

\section{Metodología de desarrollo}

Para la elaboración del proyecto se ha planteado utilizar aspectos sólidos de diversas metodologías ágiles, como SCRUM \cite{schwaber2017scrum} o \textit{Kanban} \cite{kanban}. Al estar planteadas originalmente para equipos ---generalmente pequeños---, ha sido necesario adaptarlas a un desarrollo individual. Se han seguido los principios del Manifiesto Ágil \cite{beck2001agile}.

\medskip



\medskip

Los elementos a destacar de la metodología desarrollada son los siguientes:

\subsection*{Tablero \textit{Kanban}}

Utilizado para representar los posibles estados de cada tarea de forma visual.

\medskip

Al ser \textit{Kanban} un marco flexible, se han definido estados o columnas acordes a las necesidades del proyecto. Dado que el desarrollo ha sido llevado a cabo por un individuo, se puede prescindir de estados intermedios de aprobación de código por el resto del equipo.

\medskip

Los estados definidos son los siguientes:

\begin{itemize}
	\item \textbf{TO DO}
	\item \textbf{MAKING TESTS}
	\item \textbf{IN PROGRESS}
	\item \textbf{DONE} 
\end{itemize}

El estado \textbf{MAKING TESTS} determina que que los \textit{tests} unitarios de una tarea están en desarrollo.


\subsection*{Desarrollo iterativo e incremental}

Presente en metodologías como \textit{SCRUM} o \textit{eXtreme Programming}, el desarrollo iterativo permite tener una visión temprana y real del proyecto, así como una mayor flexibilidad a la hora de detectar e implementar cambios y mejoras.

\medskip

Al tratarse de un proyecto centrado en el aprendizaje de nuevas tecnologías, esta flexibilidad ha permitido una mayor creatividad y un enfoque centrado en la implementación de la aplicación, utilizando documentación auxiliar en los casos necesarios.


\subsection*{Historias de Usuario}

Las historias de usuario


\section{Análisis de requisitos}

\section{Análisis de tecnologías}

\section{Presupuesto}


% Capítulo: diseño y resolución del trabajo
%	1. Fase de prototipado
%	2. Sistema de objetivos
%	3. Sistema de horarios
%	...
%	n-1. Accesibilidad
%	n. Producción
\chapter{Diseño y resolución del trabajo}

\section{Fase de prototipado}

\section{Sistema de objetivos}

\section{Sistema de horarios}

\section{Accesibilidad}

\section{Producción}


% Capítulo: conclusiones y vías futuras
%	1. Conclusiones
%	2. Posibles modificaciones futuras
\chapter{Conclusiones y vías futuras}

\section{Conclusiones del proyecto}

\section{Posibles modificaciones futuras}

\newpage

% Capítulo: bibliografía
%	1. Tecnologías utilizadas
%	2. Artículos consultados
\bibliographystyle{apalike}
\addcontentsline{toc}{chapter}{Bibliografía}
\bibliography{bibliography}







\end{document}
