\documentclass[10pt, a4paper]{aqademic}

\usepackage[spanish]{babel}
\selectlanguage{spanish}

\usepackage[default]{sourcesanspro}
\usepackage[T1]{fontenc}

% Document packages

\usepackage{amsmath}
\usepackage{amsfonts}
%\usepackage[type=CC, modifier=by-nc-sa, version=4.0]{doclicense}
\usepackage{graphicx}
\graphicspath{{img/}}
\usepackage[numbers]{natbib}

% Document settings

\author{Juan Ocaña Valenzuela}
\title{Trabajo Fin de Grado}

\AqSetChapter{Capítulo}

% Document composition

\begin{document}

\AqMaketitle[%
	cover    = img/logo.png,
	org      = Grado en Ingeniería Informática,
	subtitle = Aplicación de gestión de rutinas semanales basada en tecnologías web
	url      = https://github.com/patchispatch/
]
\tableofcontents

% Capítulo: introducción
%	1. Resumen
%	2. Resumen en inglés (mínimo 500 palabras)
\chapter{Introducción}

\section{Resumen}

Existe una gran oferta en cuanto a herramientas de organización personal. Desde activas comunidades que apuestan por formatos tradicionales como agendas y calendarios personalizados hasta numerosos servicios preparados para ajustarse a cualquier tipo de persona, escoger un método de organización puede suponer una odisea para alguien interesado en la materia. 

\medskip

Muchas de las herramientas más populares en este campo optan por ofrecer un servicio de suscripción ---a menudo con versión gratuita--- con ventajas tales como sincronización en la nube o integración con servicios adicionales. Sin embargo, a veces estas opciones son demasiado complejas para algunas personas, que simplemente buscan algo más sencillo o acorde a sus necesidades.

\medskip

Se plantea una propuesta de aplicación con una funcionalidad simple, orientada a reforzar y asistir en la creación de rutinas y hábitos, utilizando como soporte un sistema de objetivos y una estructura de horario semanal. La idea principal es brindar una funcionalidad simple a la vez que intuitiva, haciendo énfasis en la accesibilidad y facilidad de uso, optando por no incluir herramientas complejas, tales como un sistema de calendario o un bloc de notas ---opciones que muchos de los servicios mencionados anteriormente implementan---.

\medskip

La aplicación propuesta se ha implementado utilizando tecnologías web muy extendidas y con gran soporte en la actualidad, y se acoge a los estándares de accesibilidad WAI-ARIA (aquí va una referencia pero aún no sé cómo meterlas) definidos por el \textit{World Wide Web Consortium}. 




\section{\textit{Abstract}}


% Capítulo: estado del arte
%	1. Crítica al estado del arte
%	2. Propuesta
\chapter{Estado del arte}

\section{Crítica al estado del arte}

\section{Propuesta}


% Capítulo: análisis de objetivos y metodología
%	1. Metodología de desarrollo
%	2. Análisis de requisitos
%	3. Análisis de tecnologías
%	4. Presupuesto
\chapter{Análisis de objetivos y metodología}

\section{Metodología de desarrollo}

\section{Análisis de requisitos}

\section{Análisis de tecnologías}

\section{Presupuesto}


% Capítulo: diseño y resolución del trabajo
%	1. Fase de prototipado
%	2. Sistema de objetivos
%	3. Sistema de horarios
%	...
%	n-1. Accesibilidad
%	n. Producción
\chapter{Diseño y resolución del trabajo}

\section{Fase de prototipado}

\section{Sistema de objetivos}

\section{Sistema de horarios}

\section{Accesibilidad}

\section{Producción}


% Capítulo: conclusiones y vías futuras
%	1. Conclusiones
%	2. Posibles modificaciones futuras
\chapter{Conclusiones y vías futuras}

\section{Conclusiones del proyecto}

\section{Posibles modificaciones futuras}


% Capítulo: bibliografía
%	1. Tecnologías utilizadas
%	2. Artículos consultados
\chapter{Bibliografía}

\section{Referencias}

\section{Documentación}

\section{Miscelánea}






\end{document}
