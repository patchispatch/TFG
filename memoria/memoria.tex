\documentclass[10pt, a4paper]{aqademic}

\usepackage[spanish]{babel}
\selectlanguage{spanish}

\usepackage[default]{sourcesanspro}
\usepackage[T1]{fontenc}

% Document packages

\usepackage{amsmath}
\usepackage{amsfonts}
%\usepackage[type=CC, modifier=by-nc-sa, version=4.0]{doclicense}
\usepackage{graphicx}
\graphicspath{{img/}}
\usepackage[numbers]{natbib}
\usepackage{enumitem}
\usepackage{colortbl}

% Document settings

% Nest enumeration as numbers
\renewcommand{\labelenumii}{\theenumii}
\renewcommand{\theenumii}{\theenumi.\arabic{enumii}.}

\author{Juan Ocaña Valenzuela}
\title{Trabajo Fin de Grado}

\AqSetChapter{Capítulo}

% Document composition

\begin{document}

\AqMaketitle[%
	cover    = img/logo.png,
	org      = Grado en Ingeniería Informática,
	subtitle = Aplicación de gestión de rutinas semanales basada en tecnologías web
]
\tableofcontents

% Capítulo: introducción
\chapter{Introducción}

\textbf{Palabras clave:} palabras clave

\section{Resumen}

\textbf{NOTA: provisional}

Existe una gran oferta en cuanto a herramientas de organización personal. Desde activas comunidades que apuestan por formatos tradicionales como agendas y calendarios personalizados hasta numerosos servicios preparados para ajustarse a cualquier tipo de persona, escoger un método de organización puede suponer una odisea para alguien interesado en la materia. 

\medskip

Muchas de las herramientas más populares en este campo optan por ofrecer un servicio de suscripción ---a menudo con versión gratuita---, que aporta ventajas tales como la sincronización de los datos del usuario en la nube o integración con servicios adicionales. Sin embargo, a veces estas opciones son demasiado complejas para algunas personas, que simplemente buscan algo más sencillo o acorde a sus necesidades.

\medskip

Se plantea una propuesta de aplicación con una funcionalidad simple, orientada a asistir en la creación de rutinas y hábitos, utilizando como soporte un sistema de objetivos y una estructura de horario semanal. La idea principal es brindar una funcionalidad simple a la vez que intuitiva, haciendo énfasis en la facilidad de uso, optando por no incluir herramientas complejas, tales como un sistema de calendario o un bloc de notas ---opciones que muchos de los servicios mencionados anteriormente implementan---.

\medskip

La aplicación propuesta se ha desarrollado utilizando tecnologías web extendidas y con gran soporte en la actualidad, y una arquitectura cliente-servidor basada en el patrón \textit{REST} \cite{fielding2000architectural}.

\newpage


\section{\textit{Abstract}}

Una vez esté aprobado el resumen, traducir.


\newpage

\section{Integración de conocimientos previos}

Para la realización de este proyecto, han sido útiles los conocimientos adquiridos en las siguientes asignaturas del grado:

\begin{itemize}
	\item Fundamentos de Bases de Datos
	\item Fundamentos de Ingeniería del Software
	\item Ingeniería de Servidores
	\item Sistemas de Información Basados en Web
	\item Dirección y Gestión de Proyectos
	\item Metodologías de Desarrollo Ágil
\end{itemize}

% Capítulo: estado del arte
\chapter{Estado del arte}

\section{Crítica al estado del arte}

El mercado de aplicaciones de organización personal está repleto de opciones, con algunas de ellas muy populares. Las aplicaciones estudiadas pueden dividirse en las siguientes categorías, según su principal enfoque en cuanto a la organización personal

\subsection*{Aplicaciones de agenda y calendario}

Las aplicaciones de agenda y calendario son las más populares, debido a su enfoque principalmente profesional. Se trata de herramientas completas y potentes, con capacidad para gestionar diferentes calendarios simultáneos, recordatorios, exportación a distintos formatos e integración con otros servicios ---como por ejemplo es el caso de Google Calendar, en el que se pueden planificar citas a través de \textit{Gmail}, o adjuntar salas de videoconferencia de \textit{Google Meet}---. 

\medskip

Una de las principales ventajas de los principales servicios de este tipo es su gratuidad. Al estar mantenidos en su mayoría por grandes empresas tecnológicas ---Google, Microsoft, Apple, etc.---, no requieren de un esfuerzo adicional para el usuario del resto de sus respectivas aplicaciones.

\medskip

Estos servicios, pese a estar preparados para un uso intensivo y profesional, también se adaptan al usuario medio en mayor o menor medida. No obstante, su complejidad puede resultar abrumadora, y ante todo, suplen necesidades específicas: una agenda y un calendario tradicionales.

\medskip

Algunas de las aplicaciones de calendario más populares o influyentes son las siguientes:

\begin{itemize}
	\item Google Calendar
	\item Microsoft Outlook
	\item Apple Calendar (anteriormente llamado \textit{iCal})
\end{itemize}


\subsection*{Aplicaciones de tareas}

Otro enfoque popular a la hora de ofrecer herramientas de organización personal es el de las aplicaciones de tareas. Estos servicios ofrecen la posibilidad de planificarse a través de listas de tareas, a menudo con añadidos como notificaciones y recordatorios, similares a los presentes en las aplicaciones más tradicionales anteriormente mencionadas. 

\medskip

Estas aplicaciones utilizan la \textbf{tarea} como elemento base, construyendo todo su funcionamiento en torno a ella, y poniendo el foco en facilitar la interacción del usuario con la misma. Pese a estar presentes en entornos profesionales, muchas de estas aplicaciones se orientan hacia el usuario medio en primer lugar, ofreciendo una experiencia más sencilla e intuitiva.

\medskip

No obstante, muchos de estos servicios cuentan con una funcionalidad limitada, a menudo como expositor de una suscripción; y aquellos que no disponen de esta opción suelen requerir de un esfuerzo adicional por parte del usuario ---por ejemplo, Tasks.org \cite{tasks.org}---.

\medskip

Algunos ejemplos de aplicaciones de tareas son las siguientes:

\begin{itemize}
	\item any.do
	\item Todoist
	\item Tasks.org
\end{itemize}


\subsection*{Aplicaciones de espacio de trabajo}

Existen aplicaciones cuya premisa es ofrecer al usuario diversas herramientas para crear su método de organización, con todas las ventajas de un soporte digital. Estas aplicaciones suelen brindar al usuario sistemas de bases de datos sencillos y fáciles de manipular, con tal de organizar notas y documentos compuestos de distintos módulos preestablecidos. No existe un estándar para las aplicaciones de este tipo, y cada servicio define su propio sistema ---por ejemplo, \textit{Notion} centra sus esfuerzos en las bases de datos, mientras que \textit{Evernote} lo hace en un completo sistema de notas---.

\medskip

Pese a que este tipo de aplicaciones ofrecen una gran versatilidad, requieren un esfuerzo adicional por parte del usuario para comprender y aprender a utilizar las herramientas a su disposición. 

\medskip

Algunas aplicaciones de este tipo son las siguientes:

\begin{itemize}
	\item Notion
	\item Evernote
	\item Microsoft OneNote
\end{itemize}

\subsection*{Aplicaciones de seguimiento de hábitos}

Por último, existe una categoría de aplicaciones de organización enfocadas al seguimiento de diferentes objetivos, ya sean definidos por el usuario o preestablecidos en base a unas pautas concretas. 

\medskip

Dichas aplicaciones, mayoritariamente publicadas para dispositivos móviles, permiten registrar los avances del usuario en diferentes tareas, contabilizar el progreso y presentarlo en distintos formatos. Muchas de estas aplicaciones implementan sistemas de ludificación \cite{inproceedings} ---también llamada \textit{gamificación}---, utilizando logros, puntos y otros mecanismos similares para mantener el interés del usuario.

\medskip

Un aspecto relevante de estos servicios es que suelen centrarse en temáticas específicas, como el deporte, hábitos saludables, productividad o registro del estado de ánimo, por poner algunos ejemplos. Casi todos ellos siguen orientados a construir rutinas, ya sean diarias, como mantener un horario de sueño constante, hidratarse o regar las plantas, o con diferentes plazos ---semanales, mensuales, personalizados, etc.---, pero la forma de presentar la información al usuario es muy diferente. 

\medskip

Debido a las grandes diferencias entre las aplicaciones de seguimiento de hábitos estudiadas, es difícil definir un usuario objetivo claro. Están orientadas a un perfil personal, pero muchas de ellas tratan de implementar soluciones a problemas demasiado específicos y abusan de la ludificación, volviendo su uso tedioso.

\medskip

Algunas aplicaciones populares en este ámbito son las siguientes:

\begin{itemize}
	\item Rabit
	\item Dailyo
	\item Habit (\textit{Leap Fitness Group})
	\item Habitica
\end{itemize}


\subsection*{Crítica}

No faltan soluciones de este tipo en el mercado. Desde un ámbito más profesional hasta el usuario más ajeno a la tecnología, existe una gran variedad de aplicaciones para organizarse, con diversos enfoques y objetivos. 

Pese a lo completas y maduras que son las aplicaciones más tradicionales, su enfoque profesional y multitud de funcionalidades pueden suponer un obstáculo para un perfil de usuario que no sólo no va a beneficiarse de todas las integraciones y ventajas que ofrecen, sino que no las necesita. 

\medskip

Las aplicaciones de tareas suplen este exceso de complejidad con un planteamiento básico y suficiente, añadiendo funciones complementarias, a menudo similares a las de servicios tradicionales, pero adaptadas al flujo de la lista de tareas. No obstante, muchas de estas funciones requieren de una suscripción, estando las aplicaciones muy limitadas en su versión gratuita al no disponer de una infraestructura tan grande. 

\medskip

Las aplicaciones de espacio de trabajo son herramientas muy potentes, que en manos de usuarios experimentados pueden resultar muy útiles y servir multitud de propósitos. Muchas de ellas ofrecen plantillas predefinidas para usos comunes, y permiten establecer sistemas de organización a medida, pero suelen centrarse en colecciones de datos en lugar de eventos y tareas, y aunque puedan definirse fechas, notificaciones y, en algunos casos, calendarios, suelen ser atributos secundarios, que necesitan trabajo previo por parte del usuario para funcionar.

\medskip

Por último, las aplicaciones de seguimiento de hábitos siguen un propósito específico, alejado del resto de sistemas. Su especificidad hace que este tipo de aplicaciones sean fáciles de entender con respecto a otras soluciones. No requieren de gran configuración previa, su función es limitada pero concreta, y sirven como apoyo al usuario para conseguir sus objetivos y ver su progreso de forma sencilla, independientemente de los medios que utilicen para llevarlos a cabo. No pretenden ser un sustituto de otros servicios como las aplicaciones de calendario o espacios de trabajo, y pueden llegar a complementarse muy bien.

No obstante, pese a que gozan de gran popularidad y variedad, muchas de ellas ofrecen incentivos y mecanismos tan distintos al usuario que pueden resultar desde poco interesantes a abrumadoras. Además, el soporte de muchas de estas aplicaciones comparadas con el resto de casos estudiados suele ser escaso, o depender de servicios externos a configurar por el usuario. Esto último no es necesariamente algo negativo, pero puede inclinar al usuario a explorar otras opciones.




\section{Propuesta}

\subsection{Objetivos}

Se plantean los siguientes objetivos para la aplicación propuesta. Pueden dividirse en varias secciones, cada una con sus propios objetivos anidados de diferente prioridad:

\begin{itemize}
	\item La aplicación debe permitir registrar, modificar y eliminar objetivos definidos por el usuario, permitiendo una visualización clara y detallada de su progreso, e incentivar el seguimiento de los mismos.
	
	\item La aplicación debe permitir registrar, modificar y eliminar actividades habituales que realiza el usuario, y mostrar un horario semanal con las mismas, de forma que el usuario pueda visualizar de forma clara su tiempo.
	
	\item La aplicación debe permitir registrar, modificar y eliminar categorías definidas por el usuario, para que pueda organizar sus objetivos y actividades de forma temática, así como visualizar su progreso de la misma manera.
	
	\item La aplicación debe mostrar un panel de navegación entre las diferentes vistas, con una lista de categorías definidas por el usuario, un indicador de fecha y hora, y acceso a las opciones de configuración, así como sugerencias de objetivos a completar.
	
	\item La aplicación debe permitir configurar aspectos como el día de la semana en el que se reinician los objetivos definidos por el usuario, la vista por defecto y el tema de la aplicación, y la zona horaria.
	
	\item La aplicación debe cumplir con los niveles de conformidad WCAG... (investigar)
\end{itemize}


\subsection*{Sección de objetivos}

\subsubsection*{Fundamentales}

\begin{enumerate}
	\item Se debe poder crear, consultar, editar y eliminar objetivos.
	
	\item Se debe poder registrar una entrada de un objetivo, y ver reflejado el progreso del mismo.
	
	\item Se debe poder ver una lista de objetivos, que permita ser filtrada por estado de finalización.
	
	\item El progreso de los objetivos debe reiniciarse en el día de la semana establecido.
	
	\item Se debe mostrar una racha de finalización para cada objetivo, que indique el número de veces consecutivas en las que se ha completado. Tras completar un objetivo, se ha de incrementar dicha racha.
	
	\item Se debe guardar la máxima racha alcanzada para cada objetivo. En caso de ser superada, se actualizará debidamente.
	
	\item Se debe mostrar una vista con información general sobre los objetivos definidos, que incluya una lista de los mismos, detalles sobre las rachas, y un historial de entradas.
\end{enumerate}


\subsubsection*{Opcionales}

\begin{enumerate}[resume]
	\item Se debe permitir pausar un objetivo indefinidamente. Esto lo excluirá de la lista de objetivos activos, y no se podrá introducir nuevas entradas del mismo hasta que no sea reanudado. La racha actual se guardará.
	
	\item Se debe poder definir diferentes frecuencias al crear objetivos, que determinarán el reinicio del progreso de los mismos. Se podrá seleccionar entre frecuencia semanal y mensual.
\end{enumerate}


\subsection*{Sección de actividades}

\subsubsection*{Fundamentales}

\begin{enumerate}[resume]
	\item Se debe poder crear, consultar, editar y eliminar actividades.
	
	\item Se debe mostrar una vista con el horario de la semana en formato de lista de actividades.
\end{enumerate}

\subsubsection*{Opcionales}

\begin{enumerate}[resume]
	\item Se debe poder ver una lista de las actividades pertenecientes a una categoría en la vista de resumen de la misma.
\end{enumerate}


\subsection*{Sección de categorías}

\subsubsection*{Fundamentales}

\begin{enumerate}[resume]
	\item Se debe poder crear, consultar, editar y eliminar categorías.
	
	\item Se debe poder asignar una categoría a objetivos y actividades.
	
	\item Se debe mostrar una vista resumen de cada categoría, similar a la vista general de objetivos, con información relevante sobre los objetivos de la misma.
\end{enumerate}


\subsection*{Sección de panel de navegación o \textit{Vista Rápida}}

\subsubsection*{Fundamentales}

\begin{enumerate}[resume]
	\item Se debe mostrar la fecha y hora actuales.
	
	\item Se deben mostrar varios objetivos pendientes de finalización como sugerencias al usuario, para incentivarlo a completarlos en caso de no saber qué hacer.
	
	\item Se debe mostrar un menú de navegación con las diferentes vistas principales de la aplicación para poder acceder a ellas.
	
	\item Se debe mostrar un acceso a la configuración de la aplicación.
\end{enumerate}

\subsubsection*{Opcionales}

\begin{enumerate}[resume]
	\item Se debe poder ocultar y expandir el panel de navegación.
\end{enumerate}


\subsection*{Sección de configuración}

\subsubsection*{Fundamentales}

\begin{enumerate}[resume]
	\item Se debe poder ajustar el día de la semana y la hora en la que se reinicia el progreso de los objetivos.
	
	\item Se debe poder definir la zona horaria de la aplicación.
	
	\item Se debe poder seleccionar la vista por defecto al entrar en la aplicación.
\end{enumerate}

\subsubsection*{Opcionales}

\begin{enumerate}[resume]
	\item Se debe poder seleccionar el tema de la aplicación.
	
	\item Se debe poder acceder a la configuración de las categorías desde los ajustes generales.
\end{enumerate}


\subsection*{Sección de accesibilidad}

\subsubsection*{Fundamentales}

\begin{enumerate}[resume]
	\item Se debe garantizar la navegabilidad de la aplicación utilizando lectores de pantalla.
\end{enumerate}


\chapter{Metodología}

\section{Metodología de desarrollo}

\subsection{Elementos utilizados}

Para la elaboración del proyecto se ha planteado utilizar aspectos sólidos de diversas metodologías ágiles, como \textit{SCRUM} \cite{schwaber2017scrum} o \textit{Kanban} \cite{kanban}. Al estar planteadas originalmente para equipos ---generalmente pequeños---, ha sido necesario adaptarlas a un modelo de desarrollo individual. Se han seguido los principios del Manifiesto Ágil \cite{beck2001agile}.

\medskip

Los elementos a destacar de la metodología desarrollada son los siguientes:

\subsection*{Tablero \textit{Kanban}}

Utilizado para representar los posibles estados de cada tarea de forma visual.

\medskip

Al ser \textit{Kanban} un marco flexible, se han definido estados o columnas acordes a las necesidades del proyecto. Dado que el desarrollo ha sido llevado a cabo por un individuo, se puede prescindir de estados intermedios, como por ejemplo <<pendiente de aprobación por el resto del equipo>>.

\medskip

Los estados definidos para este proyecto son los siguientes:

\begin{itemize}
	\item \textbf{PENDIENTE}
	\item \textbf{DISEÑO}
	\item \textbf{ELABORACIÓN DE TESTS}
	\item \textbf{IMPLEMENTACIÓN}
	\item \textbf{HECHO}
\end{itemize}


\subsection*{Desarrollo iterativo e incremental}

Presente en metodologías como \textit{SCRUM} o \textit{eXtreme Programming}, el desarrollo iterativo permite tener una visión temprana y real del proyecto, así como una mayor flexibilidad a la hora de detectar e implementar cambios y mejoras.

\medskip

Al tratarse de un proyecto centrado en el aprendizaje de nuevas tecnologías, esta flexibilidad ha permitido una mayor creatividad y un enfoque centrado en la implementación de la aplicación, utilizando documentación auxiliar en los casos necesarios.


\subsection*{Historias de Usuario}

Las historias de usuario describen la funcionalidad de la aplicación desde el punto de vista del usuario final, sin detalles de implementación o conocimiento sobre la arquitectura del proyecto. Se describen según la siguiente plantilla:

\medskip

\begin{figure}[h]
	\centering
	\textit{Como <<usuario>>, quiero <<algo de la aplicación>> para <<cumplir un propósito>>}
\end{figure}

Dada su sencillez y la abstracción que presentan con respecto al desarrollo, las historias de usuario se utilizan con frecuencia en metodologías de desarrollo ágil, en las que el cliente suele formar parte del proceso. Además, al representar funcionalidades desde un punto de vista final, son fácilmente extensibles.

Las historias de usuario pueden agruparse en las llamadas \textit{épicas}, que permiten agregar un contexto común y estructurar el proyecto en torno a funcionalidades o necesidades globales.


\section{Análisis de requisitos}

Para este proyecto, se han definido historias de usuario basadas en los objetivos de la aplicación descritos anteriormente, distribuidas en épicas.

Dado que se ha utilizado la herramienta \textit{Jira Software} a lo largo del desarrollo, se ha asignado un identificador único a cada historia de usuario, con el formato \textit{TFG-n}.

\subsubsection*{Sección de objetivos}

\begin{itemize}[leftmargin=16mm]
	\item [\textbf{TFG-9}] Como usuario, quiero crear un objetivo para poder seguirlo.
	
	Prioridad: muy alta.
	
	\item [\textbf{TFG-10}] Como usuario, quiero editar un objetivo para corregirlo o actualizarlo.
	
	Prioridad: muy alta.
	
	\item [\textbf{TFG-11}] Como usuario, quiero actualizar el progreso de un objetivo para registrar mis hábitos.
	
	Prioridad: muy alta.
	
	\item [\textbf{TFG-12}] Como usuario, quiero eliminar un objetivo cuando lo necesite.
	
	Prioridad: muy alta.
	
	\item [\textbf{TFG-14}] Como usuario, quiero poder ver una racha de cumplimiento de cada objetivo para mantener mi motivación.
	
	Prioridad: alta.
	
	\item [\textbf{TFG-17}] Como usuario, quiero poder ver una lista de mis objetivos actuales para tenerlos en cuenta.
	
	Prioridad: alta.
	
	\item [\textbf{TFG-18}] Como usuario, quiero que el progreso de mis objetivos se reinicie una vez pasado el plazo, para poder seguirlos correctamente.
	
	Prioridad: alta.
	
	\item [\textbf{TFG-20}] Como usuario, quiero poder ver un historial de registro de mis objetivos para poder analizarlos y ver mi progreso.
	
	Prioridad: media.
	
	\item [\textbf{TFG-22}] Como usuario, quiero poder ver una página de resumen de mis objetivos para ver mi progreso general.
	
	Prioridad: media.
	
	\item [\textbf{TFG-67}] Como usuario, quiero ver el progreso de un objetivo para seguir mi esfuerzo.
	
	Prioridad: media.
	
	\item [\textbf{TFG-84}] Como usuario, quiero ver la mayor racha que he alcanzado en cada objetivo para poder tener presentes mis logros.
	
	Prioridad: media.
	
	\item [\textbf{TFG-13}] Como usuario, quiero crear un objetivo para poder seguirlo.
	
	Prioridad: baja.
	
	\item [\textbf{TFG-19}] Como usuario, quiero poder pausar objetivos indefinidamente sin eliminarlos, para poder retomarlos en un futuro sin eliminar mis datos.
	
	Prioridad: baja.
	
	\item [\textbf{TFG-21}] Como usuario, quiero ver indicadores visuales de que un objetivo se ha actualizado correctamente, o si ha habido un error, para poder actuar en consecuencia.
	
	Prioridad: baja.
\end{itemize}


\subsubsection*{Sección de actividades}

\begin{itemize}[leftmargin=16mm]
	\item [\textbf{TFG-24}] Como usuario, quiero poder crear una actividad para tenerla en cuenta al organizarme 
	
	Prioridad: muy alta.
	
	\item [\textbf{TFG-25}] Como usuario, quiero poder editar una actividad para actualizar sus condiciones.
	
	Prioridad: muy alta.
	
	\item [\textbf{TFG-26}] Como usuario, quiero poder eliminar una actividad en caso de que ya no sea de mi interés.
	
	Prioridad: muy alta.
	
	\item [\textbf{TFG-23}] Como usuario, quiero poder acceder a una vista semanal de mis actividades para poder visualizar mi tiempo.
	
	Prioridad: alta.
	
	\item [\textbf{TFG-87}] Como usuario, quiero poder añadir los momentos en los que practico una actividad para poder organizarme.
	
	Prioridad: alta.
	
	\item [\textbf{TFG-88}] Como usuario, quiero poder editar los momentos en los que practico una actividad para poder ajustarlos a mi situación.
	
	Prioridad: media.
	
	\item [\textbf{TFG-89}] Como usuario, quiero poder eliminar los momentos en los que practico una actividad para poder ajustarlos a mi situación.
	
	Prioridad: media.
	
	\item [\textbf{TFG-28}] Como usuario, quiero ver las actividades asociadas a una categoría en la vista de esa categoría para poder verlas fácilmente.
	
	Prioridad: muy baja.
	
	\item [\textbf{TFG-91}] Como usuario, quiero que la vista de actividades empiece en el día de la semana configurado como inicio de semana.
	
	Prioridad: muy baja.
\end{itemize}

\subsubsection*{Sección de categorías}

\begin{itemize}[leftmargin=16mm]
	\item [\textbf{TFG-15}] Como usuario, quiero poder agrupar mis objetivos en categorías, para poder ordenarlos y observarlos por separado.
	
	Prioridad: media.
	
	\item [\textbf{TFG-16}] Como usuario, quiero poder ver un resumen de cada categoría de objetivos, para saber mi progreso con la misma.
	
	Prioridad: media.
	
	\item [\textbf{TFG-43}] Como usuario, quiero poder crear una categoría para organizar mis actividades y objetivos.
	
	Prioridad: media.
	
	
	\item [\textbf{TFG-44}] Como usuario, quiero poder editar una categoría si es necesario.
	
	Prioridad: baja.
	
	\item [\textbf{TFG-45}] Como usuario, quiero poder asignar objetivos a una categoría, para poder ver mi progreso de forma ordenada.
	
	Prioridad: media.
	
	\item [\textbf{TFG-46}] Como usuario, quiero poder asignar actividades a una categoría, para poder ver mi semana de forma clara.
	
	Prioridad: media.
\end{itemize}

\subsubsection*{Sección de panel de navegación o \textit{Vista rápida}}

\begin{itemize}[leftmargin=16mm]
	\item [\textbf{TFG-32}] Como usuario, quiero disponer de un menú de navegación por las diferentes vistas de la aplicación para acceder a ellas cómodamente.
	
	Prioridad: alta.
	
	\item [\textbf{TFG-31}] Como usuario, quiero ver un menú con las categorías que he creado, para poder acceder a la vista de categoría de cada una.
	
	Prioridad: media.
	
	\item [\textbf{TFG-29}] Como usuario, quiero ver la fecha y hora actual para poder tenerla presente.
	
	Prioridad: baja.
	
	\item [\textbf{TFG-30}] Como usuario, quiero ver algunos objetivos pendientes que podría completar hoy, como sugerencia por si no se me ocurre qué hacer ahora.
	
	Prioridad: baja.
	
	\item [\textbf{TFG-81}] Como usuario, quiero poder ocultar el panel lateral para poder ver con mayor tamaño el resto de la aplicación.
	
	Prioridad: muy baja.
\end{itemize}

\subsubsection*{Sección de configuración}

\begin{itemize}[leftmargin=16mm]
	\item [\textbf{TFG-34}] Como usuario, quiero poder ajustar qué día y hora se reinician mis objetivos para establecer mis rutinas.
	
	Prioridad: alta.
	
	\item [\textbf{TFG-33}] Como usuario, quiero poder acceder a la configuración general de la aplicación desde cualquier parte de la misma, para poder ajustar mi experiencia de forma cómoda.
	
	Prioridad: media.
	
	\item [\textbf{TFG-36}] Como usuario, quiero poder acceder a la configuración de mis categorías desde la configuración general para poder tenerlo todo en el mismo sitio
	
	Prioridad: baja.
	
	\item [\textbf{TFG-37}] Como usuario, quiero poder fijar una vista por defecto para que al abrir la aplicación se muestre.
	
	Prioridad: baja.
	
	\item [\textbf{TFG-35}] Como usuario, quiero poder establecer mi zona horaria para que el reloj y el progreso funcionen adecuadamente.

	Prioridad: muy baja.
	
	\item [\textbf{TFG-37}] Como usuario, quiero poder seleccionar el tema de la aplicación para acomodarla a mis gustos y necesidades 
	
	Prioridad: muy baja.
\end{itemize}

\subsubsection*{Sección de accesibilidad}

\begin{itemize}[leftmargin=16mm]
	\item [\textbf{TFG-40}] Como usuario con problemas de visión, quiero que la aplicación sea navegable con lectores de pantalla para poder usarla cómodamente.
	
	Prioridad: muy baja. 
\end{itemize}


\section{Planificación temporal}

\subsection{Planificación inicial}

La planificación que se hubiese seguido en condiciones ideales

\subsection{Planificación final}

Debido a diversos factores, principalmente externos al proyecto, la planificación inicial no pudo llevarse a cabo. Se analizó el estado del proyecto, y se decidió reiniciar la fase de desarrollo con una nueva planificación, ajustada esta vez a una fecha de entrega posterior.

\subsubsection*{Distribución de etapas del proyecto en el tiempo}

Con tal de evitar los posibles imprevistos ---algo que, como ya se ha podido observar, ha afectado con severidad al proyecto anteriormente---, la nueva planificación se ha realizado de forma exhaustiva, presentando un mayor número de iteraciones, ajustadas de antemano con un listado de funcionalidades concretas a implementar.

Las etapas del proyecto quedaron distribuidas de la siguiente manera:

\begin{enumerate}
	\item Investigación previa: elección de tecnologías a utilizar, revisión del estado actual del proyecto y estimación del coste de los cambios necesarios.
	
	\textbf{Plazo:} del 1 al 31 de mayo. 
	
	\item Configuración inicial del proyecto: preparación de las tecnologías elegidas y todo lo necesario para comenzar a implementar la aplicación.
	
	\textbf{Plazo:} del 22 de mayo al 7 de junio.
	
	\item Iteraciones de desarrollo o \textit{sprints}: 
	\begin{enumerate}
		\item Iteración 1: del 8 al 22 de junio.
		
		\item Iteración 2: del 23 de junio al  7 de julio.
		
		\item Iteración 3: del 7 al 21 de julio.
		
		\item Iteración 4: del 22 de julio al 5 de agosto.
		
		\item Iteración 5: del 6 al 20 de agosto.
		
		\item Iteración 6:del 21 al 31 de agosto.
	\end{enumerate}
\end{enumerate}

La distribución de las etapas en el tiempo se puede consultar en el siguiente diagrama de Gantt:

\medskip

\begin{figure}[h]
	\centering
	\includegraphics[scale=0.35]{img/gantt\ sprints.png}
	\caption{Diagrama de Gantt de las etapas del proyecto}
\end{figure}


\subsubsection*{Distribución de las tareas a realizar en las etapas propuestas}

A la hora de distribuir las tareas en las distintas iteraciones, se realizó una estimación de la complejidad y relevancia de cada una de las épicas del proyecto, y se planteó su implementación de forma mayormente secuencial. No obstante, la dependencia entre algunas funcionalidades obligó a adelantar el desarrollo de historias de usuario pertenecientes a épicas menos prioritarias. La última iteración se ha dedicado a realizar ajustes menores, por lo que presenta tareas de todas las épicas.

La distribución aproximada de las épicas en las iteraciones de desarrollo se puede ver en este diagrama de Gantt:

\medskip

\begin{figure}[h]
	\centering
	\includegraphics[scale=0.35]{img/gantt\ epicas.png}
	\caption{Diagrama de Gantt de las épicas del proyecto}
\end{figure}


\subsubsection*{Distribución final de las historias de usuario en las iteraciones de desarrollo}

\textbf{Iteración 1}
\begin{itemize}[leftmargin=16mm]
	\item [\textbf{TFG-9}] Como usuario, quiero crear un objetivo para poder seguirlo.
	
	\item [\textbf{TFG-10}] Como usuario, quiero poder editar un objetivo.
	
	\item [\textbf{TFG-11}] Como usuario, quiero poder actualizar el progreso de un objetivo para registrar mis hábitos.
	
	\item [\textbf{TFG-12}] Como usuario, quiero poder eliminar un objetivo de forma permanente.
	
	\item [\textbf{TFG-15}] Como usuario, quiero poder agrupar mis objetivos en categorías, para poder ordenarlos y observarlos por separado.
	
	\item [\textbf{TFG-17}] Como usuario, quiero poder ver una lista de mis objetivos actuales para tenerlos en cuenta.
	
	\item [\textbf{TFG-18}] Como usuario, quiero que el progreso de mis objetivos se reinicie una vez pasado el plazo, para poder seguirlos correctamente.

	
	\item [\textbf{TFG-19}] Como usuario, quiero poder pausar objetivos indefinidamente sin eliminarlos, para poder retomarlos en un futuro sin eliminar mis datos.
	
	\item [\textbf{TFG-67}] Como usuario, quiero ver el progreso de un objetivo para seguir mi esfuerzo.
\end{itemize}

\textbf{Iteración 2}
\begin{itemize}[leftmargin=16mm]
	\item [\textbf{TFG-14}] Como usuario, quiero poder ver una racha de cumplimiento de cada objetivo, para mantener mi motivación.

	\item [\textbf{TFG-20}] Como usuario, quiero poder ver un historial de registro de mis objetivos para poder analizarlos y ver mi progreso.
	
	\item [\textbf{TFG-21}] Como usuario, quiero ver indicadores visuales de que un objetivo se ha actualizado correctamente, o si ha habido un error, para poder actuar en consecuencia.
	
	\item [\textbf{TFG-22}] Como usuario, quiero poder ver un resumen global de mis objetivos para ver mi progreso general.
	
	\item [\textbf{TFG-24}] Como usuario, quiero poder crear una actividad para tenerla en cuenta al organizarme.
	
	\item [\textbf{TFG-25}] Como usuario, quiero poder editar una actividad para actualizar sus condiciones.
	
	\item [\textbf{TFG-26}] Como usuario, quiero poder eliminar una actividad en caso de que ya no sea de mi interés.
	
	\item [\textbf{TFG-84}] Como usuario, quiero ver la mayor racha que he alcanzado en cada objetivo para poder tener presentes mis logros.
\end{itemize}

\textbf{Iteración 3}
\begin{itemize}[leftmargin=16mm]
	\item [\textbf{TFG-16}] Como usuario, quiero poder ver un resumen de cada categoría de objetivos, para saber mi progreso con la misma.
	
	\item [\textbf{TFG-23}] Como usuario, quiero poder acceder a una vista semanal de mis actividades para poder visualizar mi tiempo.
	
	\item [\textbf{TFG-43}] Como usuario, quiero poder crear una categoría para organizar mis actividades y objetivos.
	
	\item [\textbf{TFG-44}] Como usuario, quiero poder editar una categoría si es necesario.
	
	\item [\textbf{TFG-45}] Como usuario, quiero poder asignar objetivos a una categoría, para poder ver mi progreso de forma ordenada.
	
	\item [\textbf{TFG-46}] Como usuario, quiero poder asignar actividades a una categoría, para poder ver mi semana de forma clara.
	
	\item [\textbf{TFG-87}] Como usuario, quiero poder añadir los momentos en los que practico una actividad para poder organizarme.
	
	
	\item [\textbf{TFG-88}] Como usuario, quiero poder editar los momentos en los que practico una actividad para poder ajustarlos a mi situación.
	
	\item [\textbf{TFG-89}] Como usuario, quiero poder eliminar los momentos en los que practico una actividad para poder ajustarlos a mi situación.
\end{itemize}

\textbf{Iteración 4}
\begin{itemize}[leftmargin=16mm]
	\item [\textbf{TFG-29}] Como usuario, quiero ver la fecha y hora actual para poder tenerla presente .

	\item [\textbf{TFG-30}] Como usuario, quiero ver algunos objetivos pendientes que podría completar hoy, como sugerencia por si no se me ocurre qué hacer ahora.
	
	\item [\textbf{TFG-31}] Como usuario, quiero ver un menú con las categorías que he creado, para poder acceder a la vista de categoría de cada una.
	
	\item [\textbf{TFG-32}] Como usuario, quiero disponer de un menú de navegación por las diferentes vistas de la aplicación para acceder a ellas cómodamente.
	
	\item [\textbf{TFG-33}] Como usuario, quiero poder acceder a la configuración general de la aplicación desde cualquier parte de la misma, para poder ajustar mi experiencia de forma cómoda.
	
	\item [\textbf{TFG-34}] Como usuario, quiero poder ajustar qué día y hora se reinician mis objetivos para establecer mis rutinas.
\end{itemize}

\textbf{Iteración 5}
\begin{itemize}[leftmargin=16mm]
	\item [\textbf{TFG-36}] Como usuario, quiero poder acceder a la configuración de mis categorías desde la configuración general para poder tenerlo todo en el mismo sitio.

	\item [\textbf{TFG-38}] Como usuario, quiero poder fijar una vista por defecto para que al abrir la aplicación se muestre.
	
	\item [\textbf{TFG-40}] Como usuario con problemas visuales, quiero que la aplicación sea navegable con lectores de pantalla para poder usarla cómodamente .
	
	\item [\textbf{TFG-91}] Como usuario, quiero que la vista de actividades empiece en el día de la semana configurado como inicio de semana.
\end{itemize}

\textbf{Iteración 6}
\begin{itemize}[leftmargin=16mm]
	\item [\textbf{TFG-13}]  	
	
	Como usuario, quiero poder definir objetivos con varios tipos de progreso, para poder flexibilizar mis objetivos .
	
	\item [\textbf{TFG-28}] Como usuario, quiero ver las actividades asociadas a una categoría en la vista de esa categoría para poder ver fácilmente cuáles son.
	
	\item [\textbf{TFG-35}] Como usuario, quiero poder establecer mi zona horaria para que el reloj y el progreso funcionen adecuadamente.
	
	\item [\textbf{TFG-37}] Como usuario, quiero poder seleccionar el tema de la aplicación para acomodarla a mis gustos y necesidades.
	
	\item [\textbf{TFG-81}] Como usuario, quiero poder ocultar el panel lateral para poder ver con mayor tamaño el resto de la aplicación.
\end{itemize}


\section{Presupuesto}

Se asume que el encargo ha sido realizado a una empresa especializada, que dispone de su propio material y 
hace frente a los costes energéticos y logísticos necesarios. Por tanto, el presupuesto aproximado que se ofrece cubre únicamente los costes
de desarrollo basados en horas de trabajo. Tampoco se tienen en cuenta impuestos aplicados.

\subsection*{Recursos humanos}

\begin{table}[h!]
	\begin{center}
		\begin{tabular}{|l|l|l|l|}
			\hline
			\rowcolor[HTML]{EFEFEF} 
			\textbf{Servicio}                   & \textbf{Tarifa (€/mes)} & \textbf{Duración (meses)} & \textbf{Total}              \\ \hline
			Salario de ingeniero informático junior a jornada completa & 1500       & 4          & \multicolumn{1}{r|}{6.000€} \\ \hline
		\end{tabular}
	\end{center}
\end{table}

Se ofrece además un presupuesto aproximado del coste de mantenimiento de la aplicación por parte del equipo de desarrollo.

\begin{table}[h!]
	\begin{center}
		\begin{tabular}{|l|l|l|l|}
			\hline
			\rowcolor[HTML]{EFEFEF} 
			\textbf{Servicio}                   & \textbf{Tarifa (€/hora)} \\ \hline
			Salario de ingeniero informático junior a media jornada & 16   \\ \hline
		\end{tabular}
	\end{center}
\end{table}



\subsection*{Servicios opcionales de despliegue y mantenimiento}

Se ofrece además un presupuesto orientativo sobre el coste de desplegar y mantener la aplicación en \textit{Amazon Web Services}, utilizando instancias de coste moderado.

\begin{table}[h!]
	\begin{center}
		\begin{tabular}{|l|r|r|}
			\hline
			\rowcolor[HTML]{EFEFEF} 
			\textbf{Servicios opcionales}                                                                                                                                                           & \multicolumn{1}{l|}{\cellcolor[HTML]{EFEFEF}\textbf{Tarifa (€/hora)}} & \multicolumn{1}{l|}{\cellcolor[HTML]{EFEFEF}\textbf{Coste/año}} \\ \hline
			\begin{tabular}[c]{@{}l@{}}Despliegue de la base de datos en Amazon Aurora con una instancia \\ de tipo db.t2.medium, modelo Reserved, arrendamiento de un año, No Upfront\end{tabular} & 0,055                                                                 & 481,80€                                                         \\ \hline
			Despliegue de la aplicación en un servidor virtual Amazon EC2 de tipo r4.xlarge                                                                                                         & 0,053                                                                 & 464,28€                                                         \\ \hline
		\end{tabular}
	\end{center}
\end{table}

\subsection*{Total}

\begin{table}[h!]
	\begin{center}
		\begin{tabular}{|
				>{\columncolor[HTML]{EFEFEF}}l |l|}
			\hline
			\textbf{Total}                                          & 6.000€    \\ \hline
			\textbf{Total, con servicios opcionales durante un año} & 5.296,08€ \\ \hline
		\end{tabular}
	\end{center}
\end{table}


% Capítulo: diseño y resolución del trabajo
\chapter{Desarrollo del proyecto}

\section{Primera iteración}

\subsection{Análisis}

\subsection{Diseño}

\subsection{Implementación}

\subsection{Pruebas}


% Capítulo: conclusiones y vías futuras
\chapter{Conclusiones y vías futuras}

\section{Conclusiones del proyecto}

\section{Posibles modificaciones futuras}

\newpage

% Capítulo: bibliografía
\bibliographystyle{apalike}
\addcontentsline{toc}{chapter}{Bibliografía}
\bibliography{bibliography}







\end{document}
